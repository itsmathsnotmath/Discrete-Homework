\documentclass[12pt]{article}
\usepackage{amsmath, amssymb}  % Add useful math symbols and environments
\usepackage{amsfonts}          % Add fonts for sets like \mathbb{Z}
\usepackage{enumitem}          % Better control of list formatting
\usepackage[amsthm,thmmarks]{ntheorem} %Proof formatting
\newcommand{\Z}{\mathbb{Z}}    % Custom command for integers
\newcommand{\sset}{\subseteq}  % Custom command for subset notation

\begin{document}

\begin{center}
    {\LARGE Discrete Math - Homework 5} \Large \newline
    Name:
    % Write your name here.
\end{center}

\vspace{1em}

\noindent \emph{Instruction summary:} Your work must be uploaded to Gradescope as a single PDF file. It must be typed in LaTeX to avoid a 20\% penalty. The polished proof must start in a new page (\textbackslash{newpage}). It will be graded based on clarity, LaTeX prowess and proof quality (including, for example, structure and variable definition).

%------------------------------
% EXERCISES
%------------------------------
\section*{Exercises}

\begin{enumerate}[itemsep=1.2em]

% QUESTION 1
\item \emph{(2 points)} Let \( n \in \mathbb{Z} \) with \( n \geq 1 \). Show that congruence modulo \( n \) is compatible with addition and multiplication. In other words, show that for \( a, b, c, d \in \mathbb{Z} \), if \( a \equiv b \; [n] \) and \( c \equiv d \; [n] \), then
\[
(a+c) \equiv (b+d) \; [n]
\]
and
\[
ac \equiv bd \; [n].
\]\newline
    % Write your answers to question 1 here.

% QUESTION 2
\item \emph{(4 points)}

\begin{enumerate}
    \item For \( x, y \in \mathbb{Z} \), define a relation \( \mathcal{R} \) by writing \( x \mathcal{R} y \) if \( |x - y| \leq 2 \). Is \( \mathcal{R} \) reflexive? Symmetric? Antisymmetric? Transitive? For each property, show it if it is true, or give a counterexample if it is false.\newline
    % Write your answers to question 2a here.
    
    \item Same question if \( x \mathcal{R} y \) means that \( x \) and \( y \) have a common prime factor (a prime number that divides both \( x \) and \( y \)), where \( x, y \in \mathbb{Z} \).\newline
    % Write your answers to question 2b here.
    
    \item Same question with the relation
    \[
    x \mathcal{R} y \iff x \cap y \neq \emptyset
    \]
    defined for \( x, y \in 2^{\mathbb{Z}} \).\newline
    % Write your answers to question 2c here.
    
    \item Consider the set \( A = \{0, 1, 2, \dots, 8 \} \). Define a relation \( \mathcal{R} \) on \( A \) by
    \[
    a \mathcal{R} b \iff a^2 \equiv b^2 \; [9].
    \]
    Show that \( \mathcal{R} \) is an equivalence relation, then determine all its (distinct) equivalence classes. Explain. \newline
    % Write your answers to question 2d here.
    
\end{enumerate}

% QUESTION 3
\item \emph{(4 points)} Prove combinatorially that for \( k, n \in \mathbb{N} \) and \( 1 \leq k \leq n \),
\[
k \binom{n}{k} = n \binom{n-1}{k-1}.
\] \newline
    % Write your answers to question 3 here.

% QUESTION 4
\item \emph{(5 points)} Count the following objects.

\begin{enumerate}
    \item The number of anagrams (including nonsensical words) of the word MISSISSIPPI.\newline
    % Write your answers to question 4a here.
    
    \item The number of partitions into two parts of the set \( \{1, 2, \dots, 100 \} \). Remember that both parts should be non-empty. \newline
    % Write your answers to question 4b here.
    
    \item The number of lists of length \( n \), with elements in \( \{1, 2, \dots, 10\} \), where 1 appears exactly \( k \) times, where \( 1 \leq k \leq n \). \newline
    % Write your answers to question 4c here.
    
    \item The number of ways to split 40 students into 4 groups of 10. \newline
    % Write your answers to question 4d here.
    
    \item The number of ways to make a necklace with 2 blue beads, 3 red ones, and 4 white ones. \newline
    % Write your answers to question 4e here.
\end{enumerate} 


\end{enumerate}
\newpage %Please do not erase this line.


%------------------------------
% POLISHED PROOF
%------------------------------
\section*{Polished Proof} 

\emph{(10 points)} Prove combinatorially that for \( m, k, n \in \mathbb{N} \) and \( m \leq k \leq n \),
\[
\binom{n}{k} \binom{k}{m} = \binom{n}{m} \binom{n-m}{k-m}
\]

\noindent \emph{Claim.}
% Write here the statement you intent to prove. E.g: "Claim: There is only one even prime number."

\begin{proof}
%Write your proof here.
\end{proof}

\end{document}