\documentclass[11pt]{article}
\usepackage{amsmath, amssymb}  % Add useful math symbols and environments
\usepackage{amsfonts}          % Add fonts for sets like \mathbb{Z}
\usepackage{enumitem}          % Better control of list formatting
\usepackage[amsthm,thmmarks]{ntheorem} %Proof formatting
\newcommand{\Z}{\mathbb{Z}}    % Custom command for integers
\newcommand{\sset}{\subseteq}  % Custom command for subset notation

\begin{document}

\begin{center}
	{\LARGE Discrete Math - Homework 6} \Large \newline
    Name:
    % Write your name here.
\end{center}

\noindent \emph{Instruction summary:} Your work must be uploaded to Gradescope as a single PDF file. It must be typed in LaTeX to avoid a 20\% penalty. The polished proof must start in a new page (\textbackslash{newpage}). It will be graded based on clarity, LaTeX prowess and proof quality (including, for example, structure and variable definition).

%------------------------------
% EXERCISES
%------------------------------
\section*{Exercises:}

\begin{enumerate}

% QUESTION 1
\item \emph{(3 points)}

\begin{enumerate}
\item Let \(n \in \mathbb{Z}\). Show that if \(n^2 - 6n + 5\) is even, then \(n\) is odd.\newline
    % Write your answers to question 1a here.
\item Let \(a, b, c \in \mathbb{Z}\). Show that if \(a^2 + b^2 = c^2\), then \(a\) or \(b\) is even.\newline
    % Write your answers to question 1b here.
\item Let \(p \in \mathbb{Z}\). Show that if \(p\) is prime, then \(\sqrt{p}\) is irrational (that is, it is not a rational number). You can use the following two facts:
\begin{itemize}
\item Any rational number \(x > 0\) can be written \(x = \frac{a}{b}\) where \(a, b\) are positive natural numbers with no common factor \(c \geq 2\).
\item If \(q\) is prime and \(q \mid n^2\), then \(q \mid n\).
\end{itemize} 
    % Write your answers to question 1c here.
\end{enumerate}


% QUESTION 2
\item \emph{(4 points)} The goal of this exercise is to show the following variant of the well-ordering principle (WOP).
\paragraph{Theorem:}
\emph{Assume that \(B \subseteq \mathbb{Z}\) is not empty and bounded above, which means that there exists \(M \in \mathbb{Z}\) such that \(n \leq M\) for all \(n \in B\). Then \(B\) has a largest element. In other words, there exists \(n_0 \in B\) such that \(n \leq n_0\) for all \(n \in B\). This element is denoted by \(n_0 = \max B\).}

To this end, consider \(B\) and \(M\) as in the previous statement, and define the set
\[
A = \{ M - n : n \in B\}.
\]

For example, if \( B=\{-1, 5,6\}\) and \( M=10\), then \(A=\{4, 5,11\}\).

\begin{enumerate}
\item Why does the WOP apply to \(A\)? \newline
\emph{Hint: Show that \( A\) is a non-empty subset of \(\mathbb{N}\).} \newline
    % Write your answers to question 2a here.

\item Consider \(m_0 = \min A\), and conclude the proof. \newline
\emph{Hint: Use the standard WOP on A to guarantee that it has a minimum \(m_0\). Then find \(n_0\) as a function of \(m_0\). Show that it is larger than any other \(n \in \mathbb{B}\) to prove the theorem above.}\newline
    % Write your answers to question 2b here.

\end{enumerate}


% QUESTION 3
\item \emph{(4 points)} \textit{For this exercise, you will need the result of Exercise 2.}

\begin{enumerate}
\item We want to show that any integer is either even or odd. It is enough to do so for natural numbers, so let us take \(n \in \mathbb{N}\) and define the set
\[
B = \{ k \in \mathbb{N} : 2k \leq n \}.
\]
So, for example, if \( n=7 \) then \( B=\{0,1,2,3\}\); and if \(n=4\) then \( B=\{0,1,2\}\).\newline
Show that the result of Ex. 2 applies to \(B\), consider \(q = \max B\), show that \(r = n - 2q\) can only be 0 or 1, and conclude. \newline
\emph{Hint: Prove that \( r\ge 0\) and then prove by contradiction that \(r< 2\)}. \newline
    % Write your answers to question 3a here.


\item We want to show that any rational number \(x > 0\) can be written \(x = \frac{a}{b}\) where \(a, b\) are positive natural numbers with no common factor \(c \geq 2\). So let us take a rational \(x > 0\), which by definition, we can write \(x = \frac{p}{q}\) with \(p, q \in \mathbb{N}\), \(p, q \neq 0\). Consider
\[
B = \{n \in \mathbb{N} : n \mid p \wedge n \mid q \}.
\]
Show that the result of Ex. 2 applies to \(B\), consider \(n_0 = \max B\), and then show that \(a = \frac{p}{n_0}\) and \(b = \frac{q}{n_0}\) satisfy the conditions.\newline
\emph{Hint: Proof that \(a\) and \( b\) are positive integers, \(\frac{a}{b}=x\) and proof by contradiction that there is no common factor \(c \geq 2\).}
    % Write your answers to question 3b here.

\end{enumerate}


% QUESTION 4
\item \emph{(4 points)} Consider a sequence defined by \(u_0 \in \mathbb{R}\) and, for \(n \geq 0\),
\[
u_{n+1} = a u_n + b,
\]
where \(a, b \in \mathbb{R}\) and \(a \neq 1\). Then prove by induction that for all \(n \in \mathbb{N}\),
\[
u_n = a^n u_0 + b \frac{1 - a^n}{1-a}.
\] \newline
    % Write your answers to question 4 here.


\end{enumerate}
\newpage %Please do not erase this line.

%------------------------------
% POLISHED PROOF
%------------------------------
\section*{Polished proof:} 

\emph{(10 points)} Show by induction that for all \(n \in \mathbb{N}\), 
\[
\sum_{k = 0}^n k (k!) = (n+1)! - 1.
\]



\noindent \emph{Claim.}
% Write here the statement you intent to prove. E.g: "Claim: There is only one even prime number."

\begin{proof}
%Write your proof here.
\end{proof}


\end{document}