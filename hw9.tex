\documentclass[12pt]{article}
\usepackage{amsmath, amssymb}  % Add useful math symbols and environments
\usepackage{amsfonts}          % Add fonts for sets like \mathbb{Z}
\usepackage{enumitem}          % Better control of list formatting
\usepackage[amsthm,thmmarks]{ntheorem} %Proof formatting
\newcommand{\Z}{\mathbb{Z}}    % Custom command for integers
\newcommand{\sset}{\subseteq}  % Custom command for subset notation
\newcommand{\Id}{\mathrm{Id}} % Custom command for identity
\usepackage{graphicx}

\begin{document}

\begin{center}
    {\LARGE Discrete Math - Homework 9}  \Large \newline
    Name:
    % Write your name here.
\end{center}

\noindent \emph{Instruction summary:} Your work must be uploaded to Gradescope as a single PDF file. It must be typed in LaTeX to avoid a 20\% penalty. The polished proof must start in a new page (\textbackslash{newpage}). It will be graded based on clarity, LaTeX prowess and proof quality (including, for example, structure and variable definition).

%------------------------------
% EXERCISES
%------------------------------
\section*{Exercises:}

\begin{enumerate}

% QUESTION 1
\item \emph{(5 points)} In each of the following cases, describe the sample space, give its size, and compute the probability of the event \( A \).

\begin{enumerate}
\item Flip a coin four times, \( A \) is the event that there is the same number of H and T. \newline

% Write your answers to question 1a here.

\item Roll three dice, \( A \) is the event that they show the same number. \newline

% Write your answers to question 1b here.

\item Flip a coin ten times, \( A \) is the event that there is the same number of H and T. \newline

% Write your answers to question 1c here.

\item Flip a coin \( n \) times, \( A \) is the event that there are \( h \) H. \newline

% Write your answers to question 1d here.

\item Roll three dice, \( A \) is the event that the sum of the numbers showed is even. \newline

% Write your answers to question 1e here.
\end{enumerate}

% QUESTION 2
\item \emph{(3 points)} An urn contains 7 red balls \( R_1, R_2, \dots, R_7 \) and 3 blue balls \( B_1, B_2, B_3 \).

\begin{enumerate}
\item \emph{(1 point)} Assume first that you pick two balls one after the other, without replacement.

\begin{enumerate}
\item Describe the sample space, its size, and the probability of each outcome. \newline

% Write your answers to question 2a.i here.

\item Compute the probability to pick a blue ball then a red one; a red ball then a blue one; one ball of each color; two blue balls. \newline

% Write your answers to question 2a.ii here.

\end{enumerate}

\item \emph{(2 points)} Now assume that you pick all 10 balls one after the other, without replacement.

\begin{enumerate}
\item Describe the sample space, its size, and the probability of each outcome. \newline

% Write your answers to question 2b.i here.
\item Compute the probability that the first ball picked is blue, and the last one is red. \newline

% Write your answers to question 2b.ii here.
\item Compute the probability that the first and last ball picked are blue. \newline

% Write your answers to question 2b.iii here.
\item Compute the probability that you pick all red balls first, then all blue balls. \newline

% Write your answers to question 2b.iv here.
\end{enumerate}
\end{enumerate}

% QUESTION 3
\item \emph{(3 points)} Roll two dice.

\begin{enumerate}
\item What is the sample space? What is its size? What is the probability of each outcome? \newline

% Write your answers to question 3a here.

\item Consider the events \( A = \) ``the first die shows 1'' and \( B = \) ``the sum of the two dice is even''. Compute \( P(A), P(B), P(A \cap B), P(A | B), P(B | A) \). Are these events independent? Why (informally) does this make sense? \newline

% Write your answers to question 3b here.

\item Same questions for the events \( A = \) ``the sum of the dice is 10 or more'' and \( B = \) ``the first die shows an even number''. \newline

% Write your answers to question 3c here.
\end{enumerate}

% QUESTION 4
\item \emph{(4 points)} Consider the following data about breast cancer.
\begin{itemize}
\item 2\% of women have breast cancer.
\item Mammography correctly identifies 84\% of breast cancers: this means that out of all women who have cancer, 84\% receive a positive test result.
\item There are 8\% of false positives: this means that out of all women who do not have cancer, 8\% receive a positive test result.
\end{itemize}
Take a woman at random and consider the event \( C \) that she has cancer, and \( P \) that she gets a positive test result. Their complements are \( \bar{C} \) (the event that she does not have cancer) and \( \bar{P} \) (the event that she gets a negative test result).

\medskip

\textit{Show your computations, then use a calculator to write your result as a percentage rounded to one decimal place.}

\begin{enumerate}
\item Interpret the data in terms of probabilities involving the events \( C, P, \bar{C}, \bar{P} \). You should write three probabilities, one equal to 2\%, one to 84\%, and one to 8\%. No need to explain. \newline

% Write your answers to question 4a here.

\item What is the probability that a woman gets a positive test result? \newline

% Write your answers to question 4b here.

\item A woman gets a mammography and gets a positive result. What is the probability that she actually has breast cancer? Be careful to interpret this question carefully as a probability. \newline

% Write your answers to question 4c here.

\item Now, assume that a woman is in a population at risk (e.g. due to age or family history), and it is estimated that there is a 15\% chance that she has cancer prior to the test. What does it change in terms of the data given? Then assume that this woman gets a positive test, and compute the probability that she actually has cancer. Compare to the previous question and explain the difference. \newline

% Write your answers to question 4d here.
\end{enumerate}


\end{enumerate}
\newpage %Please do not erase this line.

%------------------------------
% POLISHED PROOF
%------------------------------
\section*{Polished proof:} 

\emph{(10 points)} Suppose that \( (S,P_1) \) and \( (S,P_2) \) are two sample spaces that have the same set of outcomes, \( S \). Prove by contradiction that if there exists an outcome \( s \in S \) such that \( P_1(s) < P_2(s) \), then there must exist another outcome \( t \in S \) such that \( P_1(t) > P_2(t) \).

\noindent \emph{Claim.}
% Write here the statement you intent to prove. E.g: "Claim: There is only one even prime number."

\begin{proof}
%Write your proof here.
\end{proof}

\end{document}