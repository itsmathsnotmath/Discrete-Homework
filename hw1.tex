\documentclass[12pt]{article}
\usepackage{amsmath, amssymb}  % Add useful math symbols and environments
\usepackage{amsfonts}          % Add fonts for sets like \mathbb{Z}
\usepackage{enumitem}          % Better control of list formatting
\usepackage[amsthm,thmmarks]{ntheorem}


\begin{document}

\begin{center}
    \LARGE{ \textbf{Discrete Math - Homework 1}} \Large \newline
    Name:
    % Write your name here.
\end{center}

\vspace{1em}

\noindent \emph{Instruction summary:} Your work must be uploaded to Gradescope as a single PDF file. It must be typed in LaTeX to avoid a 20\% penalty. The polished proof must start in a new page (\textbackslash{newpage}). It will be graded based on clarity, LaTeX prowess and proof quality (including, for example, structure and variable definition).



%------------------------------
% EXERCISES
%------------------------------
\section*{Exercises}

\begin{enumerate}[label=\textbf{\arabic*.}, itemsep=1.2em]

% QUESTION 1
\item \emph{(5 points)} For the following statements \(A\) and \(B\), determine the logical relationships between them: \(A \implies B\), \(B \implies A\), \(A \iff B\), or neither. Briefly explain why.

\begin{enumerate}[label=\textbf{\alph*.}, itemsep=0.8em]
    \item \(A\): \(x \geq 0\), \(B\): \(x\) is the square of a real number (here, \(x\) is a real number). \newline
    % Write your answers to question 1a here.
    
    \item \(A\): \(p\) is prime, \(B\): \(p\) is divisible by 2 (here \(p\) is an integer). \newline
    % Write your answers to question 1b here.
    
    \item \(A\): \(p\) is not prime, \(B\): \(p\) is divisible by 2 (here \(p\) is an integer). \newline
    % Write your answers to question 1c here.
    
    
    \item \(A\): \(a^2 + b^2 = 10000\), \(B\): \(a\) and \(b\) have the same parity (here, \(a\) and \(b\) are integers). \newline
    % Write your answers to question 1d here.
    
    \item \(A\): \(a\) and \(b\) are both prime, \(B\): \(a + b\) is prime (here, \(a\) and \(b\) are integers). \newline
    % Write your answers to question 1e here.
    
\end{enumerate}



% QUESTION 2
\item \emph{(2 points)} Define rigorously the following terms:
\begin{enumerate}[label=\textbf{\alph*.}, itemsep=0.8em]
    \item Define the term ``perfect square''. \newline
    % Write your answers to question 2a here.
    
    \item Define the term ``consecutive perfect squares''.\newline
      % Write your answers to question 2b here.
\end{enumerate}


% QUESTION 3
\item \emph{(4 points)} Show that the difference between consecutive perfect squares is always odd. \newline
% Write your answers to question 3 here.


% QUESTION 4
\item \emph{(4 points)} Let \(n \in \mathbb{Z}\). Prove that there exist integers \(x\), \(y\) such that \(n = 12x + 8y\) if and only if \(n\) is divisible by 4. \newline
% Write your answers to question 4 here.


\end{enumerate}
\newpage %Please do not erase this line.

%------------------------------
% POLISHED PROOF
%------------------------------
\section*{Polished Proof}

\emph{(10 points)} Show that an integer \(n\) is odd if and only if \(2n + 2\) is divisible by 4. \newline

\noindent \emph{Claim.}
% Write here the statement you intent to prove. E.g: "Claim: There is only one even prime number."

\begin{proof}
%Write your proof here.
\end{proof}

\end{document}