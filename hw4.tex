\documentclass[12pt]{article}
\usepackage{amsmath, amssymb}  % Add useful math symbols and environments
\usepackage{amsfonts}          % Add fonts for sets like \mathbb{Z}
\usepackage{enumitem}          % Better control of list formatting
\usepackage[amsthm,thmmarks]{ntheorem} %Proof formatting
\newcommand{\Z}{\mathbb{Z}}    % Custom command for integers
\newcommand{\sset}{\subseteq}  % Custom command for subset notation

\begin{document}

\begin{center}
    {\LARGE Discrete Math - Homework 4} \Large \newline
    Name:
    % Write your name here.
\end{center}

\vspace{1em}

\noindent \textbf{Instruction summary:} Your work must be uploaded to Gradescope as a single PDF file. It must be typed in LaTeX to avoid a 20\% penalty. The polished proof must start on a new page (\texttt{\textbackslash newpage}). It will be graded based on clarity (2 points), LaTeX prowess (2 points), and proof quality (6 points), including structure and variable definition.

%------------------------------
% EXERCISES
%------------------------------
\section*{Exercises}

\begin{enumerate}[itemsep=1.2em]

% QUESTION 1
\item \emph{(3 points)} Prove De Morgan's law for sets, namely, for any sets \( A, B, C \),
\[
A - (B \cup C) = (A - B) \cap (A - C).
\]
The other identity \( A - (B \cap C) = (A - B) \cup (A - C) \) is also true, but you do not need to prove it. \newline
    % Write your answers to question 1 here.

% QUESTION 2
\item \emph{(3 points)} Show that for any sets \( A, B \), we have \( A \Delta B = A \cup B \) if and only if \( A \cap B = \emptyset \). \newline
    % Write your answers to question 2 here.


% QUESTION 3
\item \emph{(4 points)} Let \( A, B, C \) be sets with \( C \neq \emptyset \). Show that \( A \times C = B \times C \) if and only if \( A = B \). Why do we need the assumption \( C \neq \emptyset \)? \newline
    % Write your answers to question 3 here.

% QUESTION 4
\item \emph{(5 points)} Give a combinatorial proof of the identity
\[
2 \times 3^0 + 2 \times 3^1 + \cdots + 2 \times 3^{n-1} = 3^n - 1,
\]
where \( n \in \mathbb{N} \). \newline
    % Write your answers to question 4 here.

\end{enumerate}
\newpage %Please do not erase this line.

%------------------------------
% POLISHED PROOF
%------------------------------
\section*{Polished Proof}

\emph{(10 points)} Give a combinatorial proof of the identity
\[
1 + 3 + \cdots + (2n - 1) = n^2,
\]
where \( n \in \mathbb{N} \). (Hint: group lists according to their largest element). \newline

\noindent \emph{Claim.}
% Write here the statement you intent to prove. E.g: "Claim: There is only one even prime number."

\begin{proof}
%Write your proof here.
\end{proof}
\end{document}